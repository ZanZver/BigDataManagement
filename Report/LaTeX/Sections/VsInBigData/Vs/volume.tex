\subsection{Volume}\label{Volume}

Volume is a representation of data coming into the computer (server). It is measured in computer storage capacity \parencite{katal2013big}. 

In our case, we have data at rest (it will not change). The total storage equals to:

\begin{table}[h]
\centering
\begin{tabular}{|l|r|}
\hline
\text{file name} & \text{size} \\
\hline
\text{chat\_join\_team\_chat.csv} & 82 \, \text{KB} \\
\text{chat\_leave\_team\_chat.csv} & 67 \, \text{KB} \\
\text{chat\_mention\_team\_chat.csv} & 238 \, \text{KB} \\
\text{chat\_respond\_team\_chat.csv} & 252 \, \text{KB} \\
\text{combined-data.csv} & 162 \, \text{KB} \\
\text{ad-clicks.csv} & 826 \, \text{KB} \\
\text{buy-clicks.csv} & 135 \, \text{KB} \\
\text{game-clicks.csv} & 33.8 \, \text{MB} \\
\text{level-events.csv} & 43 \, \text{KB} \\
\text{team-assignments.csv} & 331 \, \text{KB} \\
\text{team.csv} & 8 \, \text{KB} \\
\text{user-session.csv} & 492 \, \text{KB} \\
\text{users.csv} & 137 \, \text{KB} \\
\hline
\text{Total file size} & \text{36.436 MB} \\
\hline
\end{tabular}
\caption{Files and Sizes}
\label{FileSizes}
\end{table}

In total we are dealing with 36.436 MB (or 36436 KB). This can work as a representation of big data, but in reality we would be dealing with more sophisticated files (bigger file sizes, more file types, more files, etc...)
