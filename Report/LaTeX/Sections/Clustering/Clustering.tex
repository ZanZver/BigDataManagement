\section{Clustering results on the proposed data set}\label{Clustering}

\subsection{About Clustering}

Clustering is a process of grouping items together based on their distance \parencite{likas2003global}. Lets take the image from bellow as an example. We have 3 classes (red, green and blue). Model calculates the distances between items (nodes) and classifies them based on how close are they between each other.

\begin{figure}[H]
    \includegraphics[scale=0.30]{img/Clustering/Clustering.jpg}
    \centering
    \caption{Clustering \parencite{web:Rocketloop}}
    \label{fig:clustering}
\end{figure}

Further explanation can be seen at \ref{A3}.

Code bellow showcases how we split the data. It is similar to classification.
\begin{listing}[H]
\caption{Split data function}
\begin{minted}{python}
def train_test_split(dataframe, test_ratio=0.2):
    dataframe = dataframe.withColumn('strength', col('strength').cast('double'))

    train_df, test_df = dataframe.randomSplit
                        ([1 - test_ratio, test_ratio], seed=42)

    return train_df, test_df
\end{minted}
\end{listing}

\newpage
\subsection{K-means}

\subsubsection{About}

SVM (Support Vector Machines) works by, simply put, dividing the data into categories \parencite{wang2005comparison}. This is done with "drawing the line" on the grid.

Example bellow showcases how two classes are divided by the model.
\begin{figure}[H]
    \includegraphics[scale=0.50]{img/Classification/SVM.png}
    \centering
    \caption{SVM example \parencite{web:Rocketloop}}
    \label{fig:SVM}
\end{figure}

As with decision tree, we have used same library in order to get SVM model.
\newpage
\subsubsection{Implementation}

In order to create the model, we pass in number of clusters, training and testing data to the function. In a similar way as with classification, the model is build in order to classify team strength. At the end, model is fitted and predictions are returned.

MLlib library from PySpark was used for kmeans model.
\begin{listing}[H]
\caption{Create kmeans model}
\begin{minted}{python}
def create_kmeans_model(train_df, test_df, k):
    assembler = VectorAssembler(inputCols=['strength'], outputCol='features')

    x_train = assembler.transform(train_df).select('features')
    x_test = assembler.transform(test_df).select('features')

    kmeans = KMeans(k=k, seed=42)

    model = kmeans.fit(x_train)

    predictions = model.transform(x_test)

    return predictions
\end{minted}
\end{listing}
\newpage
\subsubsection{Results}

Evaluate model (same function as with decision trees) was used in order to measure performace of the SVM. Model, test data and path where we want to save results is being passed to the function.
\begin{figure}[H]
    \includegraphics[scale=0.85]{img/Model/Classification/SVM/confusion_matrix.png}
    \centering
    \caption{SVM onfusion matrix}
    \label{fig:SVM_confusion_matrix}
\end{figure}


\newpage
\newpage
\subsection{GMM}

\subsubsection{About}

SVM (Support Vector Machines) works by, simply put, dividing the data into categories \parencite{wang2005comparison}. This is done with "drawing the line" on the grid.

Example bellow showcases how two classes are divided by the model.
\begin{figure}[H]
    \includegraphics[scale=0.50]{img/Classification/SVM.png}
    \centering
    \caption{SVM example \parencite{web:Rocketloop}}
    \label{fig:SVM}
\end{figure}

As with decision tree, we have used same library in order to get SVM model.
\newpage
\subsubsection{Implementation}

In order to create the model, we pass in number of clusters, training and testing data to the function. In a similar way as with classification, the model is build in order to classify team strength. At the end, model is fitted and predictions are returned.

MLlib library from PySpark was used for kmeans model.
\begin{listing}[H]
\caption{Create kmeans model}
\begin{minted}{python}
def create_kmeans_model(train_df, test_df, k):
    assembler = VectorAssembler(inputCols=['strength'], outputCol='features')

    x_train = assembler.transform(train_df).select('features')
    x_test = assembler.transform(test_df).select('features')

    kmeans = KMeans(k=k, seed=42)

    model = kmeans.fit(x_train)

    predictions = model.transform(x_test)

    return predictions
\end{minted}
\end{listing}
\newpage
\subsubsection{Results}

Evaluate model (same function as with decision trees) was used in order to measure performace of the SVM. Model, test data and path where we want to save results is being passed to the function.
\begin{figure}[H]
    \includegraphics[scale=0.85]{img/Model/Classification/SVM/confusion_matrix.png}
    \centering
    \caption{SVM onfusion matrix}
    \label{fig:SVM_confusion_matrix}
\end{figure}


\newpage


\subsection{Execute clustering code}
Code bellow is the main part that calls the functions in order to generate the models and evaluates the models.

As seen in results section, or models performed quite well. K-means seems to be the most efficient with about 30-34 clusters while GNN seems to be happy with 100 distributions.
\begin{listing}[H]
\caption{Execute the classification functions}
\begin{minted}{python}
train_df, test_df = train_test_split(mega_dataframe)
train_df, test_df = train_test_split(mega_dataframe)

evaluate_kmeans_model(train_df, test_df, 50, 
file_Path = file_paths_dict["clustering"] + "kmenas")

evaluate_gmm_model(train_df, test_df, 200, 
file_Path = file_paths_dict["clustering"] + "gmm")

\end{minted}
\end{listing}
