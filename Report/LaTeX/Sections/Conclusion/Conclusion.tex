\section{Conclusion}\label{Conclusion}

We interact with big data in our daily bases, as users or as researchers. Simple example would be IoT in the office. As we come to the office, thermostat is set, badge is scanned, parking spot is taken, etc... this is all of the data that we can use. We can measure when do users come in most often and set thermostat at that time for example.

With data we got, connections were made between users and their game scores. It would be better if data representation would be a bit larger. With larger dataset we could measure performance per quarter, do users tend to communicate more or less, are there users who often message each other, etc... At the moment data seems to be a bit of a stopping point not just due to size but also quality. 

At the end of the day, a representation of big data was accomplished. Core concepts were introduced, developed and showcased. 

\textit{Data is the new oil}, is said by many people. It seems to be true. Data production is not stopping, we are generating more and more data daily. Now we need to ask ourselves, what to do with the data?

Whole project can be found on GitHub \parencite{web:GitHub} with instructions on how to reproduce it.