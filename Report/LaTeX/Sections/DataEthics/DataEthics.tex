\section{Data ethics}\label{Data ethics}

When working with data, we need to keep in mind that it originated from somewhere or from someone. In our case, we would have two sensitive attributes, country and Twitter name. User could be harassed on their personal twitter based on the game they have played.

Our example above is a minor thing, compared to bigger data exploits that have happened. One of the biggest is Cambridge analytica \parencite{berghel2018malice}, where data from Facebook was used in order to effect election.

Further discussion can be found in \ref{A6}.