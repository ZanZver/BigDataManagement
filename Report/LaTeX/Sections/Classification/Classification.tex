\section{Classification results on the proposed data set}\label{Classification}
\subsection{About classification}

Classification is a process in machine learning where we categorise data \parencite{kotsiantis2006machine}. This is used in daily bases in our lives (example: email filter, spam or not).

Figure bellow showcases how people are divided into sick and healthy. Section \ref{A2} uses this as an example in order to go over 4 (main) steps that are needed.

\begin{figure}[H]
    \includegraphics[scale=0.13]{img/Classification/Binary-Classifier-Model.png}
    \centering
    \caption{Classification showcase \parencite{web:S-cubed}}
    \label{fig:classification}
\end{figure}

Code bellow showcase how the following steps were achieved. It looks similar since the idea was for code to be interchangeable regardless of the model.

As mentioned, we first split the data. Split is 80\% training and 20\% testing. For reproducibility reasons, seed is set to specific number (42). In our case, x (train/test) are the features that we use for prediction and y (train/test) being the feature we are trying to predict (what platform user is on).
\begin{listing}[H]
\caption{Split data function}
\begin{minted}{python}
def split_data(data_frame):
    indexer = StringIndexer(inputCol="platformType", 
                            outputCol="label")
    data_frame = indexer.fit(data_frame).transform(data_frame)

    split_ratio = [0.8, 0.2]
    seed = 42
    train_data, test_data = data_frame.randomSplit(split_ratio, 
                                                    seed=seed)

    x_train = train_data.select("platformType").toPandas()
    y_train = train_data.select("label").toPandas()

    x_test = test_data.select("platformType").toPandas()
    y_test = test_data.select("label").toPandas()

    return x_train, x_test, y_train, y_test
\end{minted}
\end{listing}

\newpage
\subsection{Decision tree}

\subsubsection{About}

SVM (Support Vector Machines) works by, simply put, dividing the data into categories \parencite{wang2005comparison}. This is done with "drawing the line" on the grid.

Example bellow showcases how two classes are divided by the model.
\begin{figure}[H]
    \includegraphics[scale=0.50]{img/Classification/SVM.png}
    \centering
    \caption{SVM example \parencite{web:Rocketloop}}
    \label{fig:SVM}
\end{figure}

As with decision tree, we have used same library in order to get SVM model.
\newpage
\subsubsection{Implementation}

In order to create the model, we pass in number of clusters, training and testing data to the function. In a similar way as with classification, the model is build in order to classify team strength. At the end, model is fitted and predictions are returned.

MLlib library from PySpark was used for kmeans model.
\begin{listing}[H]
\caption{Create kmeans model}
\begin{minted}{python}
def create_kmeans_model(train_df, test_df, k):
    assembler = VectorAssembler(inputCols=['strength'], outputCol='features')

    x_train = assembler.transform(train_df).select('features')
    x_test = assembler.transform(test_df).select('features')

    kmeans = KMeans(k=k, seed=42)

    model = kmeans.fit(x_train)

    predictions = model.transform(x_test)

    return predictions
\end{minted}
\end{listing}
\newpage
\subsubsection{Results}

Evaluate model (same function as with decision trees) was used in order to measure performace of the SVM. Model, test data and path where we want to save results is being passed to the function.
\begin{figure}[H]
    \includegraphics[scale=0.85]{img/Model/Classification/SVM/confusion_matrix.png}
    \centering
    \caption{SVM onfusion matrix}
    \label{fig:SVM_confusion_matrix}
\end{figure}


\newpage
\newpage
\subsection{SVM}

\subsubsection{About}

SVM (Support Vector Machines) works by, simply put, dividing the data into categories \parencite{wang2005comparison}. This is done with "drawing the line" on the grid.

Example bellow showcases how two classes are divided by the model.
\begin{figure}[H]
    \includegraphics[scale=0.50]{img/Classification/SVM.png}
    \centering
    \caption{SVM example \parencite{web:Rocketloop}}
    \label{fig:SVM}
\end{figure}

As with decision tree, we have used same library in order to get SVM model.
\newpage
\subsubsection{Implementation}

In order to create the model, we pass in number of clusters, training and testing data to the function. In a similar way as with classification, the model is build in order to classify team strength. At the end, model is fitted and predictions are returned.

MLlib library from PySpark was used for kmeans model.
\begin{listing}[H]
\caption{Create kmeans model}
\begin{minted}{python}
def create_kmeans_model(train_df, test_df, k):
    assembler = VectorAssembler(inputCols=['strength'], outputCol='features')

    x_train = assembler.transform(train_df).select('features')
    x_test = assembler.transform(test_df).select('features')

    kmeans = KMeans(k=k, seed=42)

    model = kmeans.fit(x_train)

    predictions = model.transform(x_test)

    return predictions
\end{minted}
\end{listing}
\newpage
\subsubsection{Results}

Evaluate model (same function as with decision trees) was used in order to measure performace of the SVM. Model, test data and path where we want to save results is being passed to the function.
\begin{figure}[H]
    \includegraphics[scale=0.85]{img/Model/Classification/SVM/confusion_matrix.png}
    \centering
    \caption{SVM onfusion matrix}
    \label{fig:SVM_confusion_matrix}
\end{figure}


\newpage

\subsection{Execute classification code}

Now that we have the models ready, we can call the code bellow in order to split the data and build models.

\begin{listing}[H]
\caption{Execute the classification functions}
\begin{minted}{python}
x_train, x_test, y_train, y_test = split_data(mega_dataframe)

dt_model = build_decision_tree_model(x_train, y_train)
svm_model = build_svm_model(x_train, y_train)

evaluate_model(dt_model, x_test, y_test, 
file_Path = file_paths_dict["classification"] + "DecisionTree")

evaluate_model(svm_model, x_test, y_test, 
file_Path = file_paths_dict["classification"] + "SVM")

\end{minted}
\end{listing}

Comparing classifications, we can see that our models did not perform the best. Based on the EDA, we know that mobile platform (iphone and android) are ~80\% of the devices used. Since our data is unbalanced, this can be a factor why results are skewed towards one end.
